

\section*{Prerequisites}


\begin{DoxyItemize}
\item G\+N\+U Autotools
\item Open\+G\+L 3.\+0
\item C++11 compiler (tested with G\+C\+C 4.\+8.\+3+)
\item \href{http://www.boost.org/}{\tt Boost}
\item \href{http://glew.sourceforge.net/}{\tt G\+L\+E\+W}
\item \href{https://www.libsdl.org/}{\tt S\+D\+L2}
\item \href{http://glm.g-truc.net/}{\tt G\+L\+M}
\end{DoxyItemize}

On Fedora 20 or later you can install these using a single command (as root)\+:

\begin{quote}
\$ yum install boost-\/$\ast$ glew-\/devel S\+D\+L2\+\_\+$\ast$ glm-\/devel \end{quote}


\section*{Building}

After cloning the source code or extracting a distributed archive, change to the directory where the source code is\+:

``` bash \$ autoreconf -\/i \$ ./configure \$ make ```

Alternatively, if you\textquotesingle{}d like to build the project in debug mode use\+:

\begin{quote}
\$ make C\+X\+X\+F\+L\+A\+G\+S=-\/\+D\+D\+E\+B\+U\+G \end{quote}


\section*{Running}

The build process should create a binary that can be executed as follows\+:

\begin{quote}
\$ ./src/shaderexample \end{quote}


See

\begin{quote}
\$ ./src/shaderexample --help \end{quote}


for usage instructions. 